% Title:  latex-insert
%
% Doc: SAMPLE.TEX
% Original Author:   S Luz <luzs@cs.tcd.ie>    
% Created:    19 Mar 2007           
%
% Change log:
% $Log$
% Revision 1.3  2009/03/06 22:41:47  cristian_bogdan
% layout
%
% Revision 1.2  2009/03/06 22:21:09  rosso_nero
% cristi's abstract, and some cutting down
%
% Revision 1.1  2009/03/06 22:05:25  rosso_nero
% stub for the paper. cristi, learn how to do that :-p
%


\documentclass{ecscw2007}
\usepackage{graphicx}
%%\usepackage{color}
%%\usepackage[pdfmark,bookmarksopen,colorlinks,urlcolor=red]{hyperref}
%%\usepackage{url}

%%\newcommand{\VERSION}  {$Revision$, \today}

\title{Aether-based Awareness Support for a\\Community of Amateur Programmers}
\author{Author 1, Author 2}
\affiliation{Institute 1, Country, Institute 2, Country} 
\email{author1@institute1.org, author1@institute1.org}

%%%%%% PDF setup
%%\hypersetup{pdftitle={}}
%%\hypersetup{pdfsubject={}}
%%\hypersetup{pdfkeywords={}}
%%\hypersetup{pdfauthor={S Luz (luzs@)}}
%%\hypersetup{citecolor=red}

\pagestyle{empty}

\begin{document}

\maketitle
\thispagestyle{empty}


\begin{abstract}
Aether is a framework that defines an awareness engine working over spaces made by semantic networks of objects. Aether defines generic manipulation of awareness information irrespective of the kinds of objects from the semantic network. Despite the potential power of such a generic mechanism, building and maintaining a shared semantic network in an actual collaborative work setting poses important challenges with today's software oriented towards personal computing. We therefore identified a case which presents a better opportunity for building such a semantic network for applying Aether mechanisms to provide awareness support. Our case is a multi-user programming sandbox for a community of amateur programmers. We describe our Aether implementation and present our plan for evaluating it in the setting.
\end{abstract}

\section*{Introduction} 

\cite{Bowers91}

\section*{Aether} 

\section*{Issues} 

\section*{Implementation} 
description of Parade
content and size of semantic network
object, relation, virtual relations
implementation of percolation steps, etc

\section*{Related Work} 


%\begin{figure}[htb]
%  \centering
%  \includegraphics[width=.3\linewidth]{flies}    
%  \caption{Figure caption}
%  \label{fig:figure}
%\end{figure}



\section*{Acknowledgments} 

{\footnotesize Acknowledgment. }



\bibliography{ecscw09-aether}
\bibliographystyle{ecscw2007}  

\end{document}

% Local Variables:
% TeX-master: t
% End:
